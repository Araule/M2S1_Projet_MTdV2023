\documentclass[12pt]{article}


% all packages
\usepackage[a4paper, total={6in, 9in}]{geometry} % pour que les marges soient plus petites
\usepackage[utf8]{inputenc} % pour que le fichier soit bien en utf-8
\usepackage{tabularx} % pour les tableaux

\begin{document}

% page 1 : page de présentation

% titre, membres de groupe et date
\title{\textbf{Rapport du Groupe X}\\ \vspace*{.5\baselineskip} Calculabilité}
\author{Eleve UN, Eleve DEUX, Eleve TROIS, Eleve QUATRE}
\date{17 Janvier 2023}

\maketitle

% à partir de la page 2
% propositions de parties et sous-parties
% n'hesitez pas à rajouter et supprimer selons vos besoins 
\newpage

\section{Création du gestionnaire de noms de variable}
Dans le cadre du projet collectif ... le groupe X se charge de la gestion de ...

\subsection{Fonctionnement du gestionnaire de noms de variable}
Bla.

\subsection{Conception du gestionnaire de noms de variable}
Bli.

\subsection{Défis rencontrés} % et comment ils ont été résolu

% sans numérotation, avec retour à la ligne
\paragraph{problème 1 \\}
Blo.

% sans numérotation, sans retour à la ligne
\paragraph{problème 2}
Blu. Blu. Blu. Blu. Blu. Blu. Blu. Blu. Blu. Blu. Blu. Blu. Blu. Blu. Blu. Blu. Blu. Blu. 

% avec numérotation, avec retour à la ligne
\subsubsection{problème 3}
Bof.

\subsection{Répartition du travail}
Bly.
    
\section{Gestion de la doc et du reporting}
Dans le cadre du projet collectif ... le groupe X se charge de la gestion de ...

\subsection{Rôle de gestionnaire de la doc et du reporting}
Bla.

\subsection{Défis rencontrés} % et comment ils ont été résolu

% sans numérotation, avec retour à la ligne
\paragraph{problème 1 \\}
Blo.

% sans numérotation, sans retour à la ligne
\paragraph{problème 2}
Blu. Blu. Blu. Blu. Blu. Blu. Blu. Blu. Blu. Blu. Blu. Blu. Blu. Blu. Blu. Blu. Blu. Blu. 

% avec numérotation, avec retour à la ligne
\subsubsection{problème 3}
Bof.

\subsection{Répartition du travail}
Bly.

\section{Si besoin de faire un tableau avec latex *wink wink groupe 1*}

\begin{table}[!ht]
\begin{center}
\begin{tabularx}{\columnwidth}{|l|X|X|} % ici trois colonnes avec section plus courte à gauche
 \hline
  & titre colonne 1 & titre colonne 2 \\
 \hline
 1  & item 11  & item 21  \\
\hline
\end{tabularx}
\caption{Titre du tableau 1}
\end{center}
\end{table}

\begin{table}[!ht]
\begin{center}
\begin{tabularx}{\columnwidth}{|X|>{\centering\arraybackslash}X|X|} % ici trois colonnes de même taille
 \hline
  & titre colonne 1 centré & titre colonne 2 \newline pas centré \\
 \hline
 1  & item 11  & item 21 pas centré  \\
\hline
\end{tabularx}
\caption{Titre du tableau 2}
\end{center}
\end{table}

\end{document}
